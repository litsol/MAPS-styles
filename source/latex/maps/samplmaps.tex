\documentclass[
% realfonts,	% if you own the official maps fonts
% nosubsub,	% if you need at most two levels of sectioning
% onecolumn,	% for symmetric single-column layout
% asym,		% for asymmetric single-column layout
]{maps}

% load packages. Please check whether you really need them.
\usepackage{natbib}

\renewcommand{\notesname}{Footnotes}

\setupArticle{      % Use {} if argument contains comma's!
% titel
          Title = {An example document for the Maps classfile, demonstrating its
                   various features\thanks{We thank the \TeX{} Live people for
                   providing us with a rich ready-to-run \TeX{} environment.}
                  },
   RunningTitle = An example document,
       SubTitle = with an optional subtitle,
% auteur
         Author = Anton Ulrich Thor,
          Email = a.u.thor@uu.am.dw,
        Address = Institute of Indefinite Studies\\
                  Unseen University\\
                  Ankh Morpork,
% tijdstip
         Period = voorjaar,
         Number = 36,
           Year = 2008,
           Page = 2,
% Intro
       Keywords = {Maps, classfile, sample},
       Language = english,
       Abstract = {This is a sample input file for the Maps classfile,
                   which is based on the article classfile.
                   It demonstrates various standard and non-standard features.
                   \\
                   Use of the abstract- and keywords environments is highly appreciated.
                  }  
}

\begin{document}
\maketitle
\section{Ordinary Text}

The ends  of words and sentences are marked
  by   spaces. It  doesn't matter how many
spaces    you type; one is as good as 100.  The
end of   a line counts as a space.

One   or more   blank lines denote the  end
of  a paragraph.

Footnotes\footnote{This is an example of a footnote.}  are converted
to endnotes\footnote{This is another one, with more text to it, to
see how it will wrap to the next line.}. You will need a
\verb+\theendnotes+ command to get them actually typeset. The title
of the notes section is defined by the command \verb+\notesname+.

\section{Sectioning}

The maps style defaults to unnumbered sections. If you really must,
you can restore section numbering with \emph{e.g.}
\begin{verbatim}
\setcounter{secnumdepth}{1}
\end{verbatim}
or higher numbers for more levels of numbering.

\subsection{Subsection}
This is a second-level section header. You can go down one more
level:

\subsubsection{A subsubsection}

This is a run-in header. The dot at the end of the section title is
added by the classfile.

\subsubsection{Tip}
If you only need one or two levels of header, then you can get a
better layout with the \texttt{nosubsub} document option.
The Maps editors may decide to turn on this option for you.

\section{Lists}

Another frequently-displayed structure is a list.
The following is an example of an \emph{itemized}
list.
\begin{itemize}
  \item This is the first item of an itemized list.
    Each item in the list is marked with a ``tick''.
  \item This is the second item of the list.  It
    contains another list nested inside it.  The inner
    list is an \emph{enumerated} list.
    \begin{enumerate}
      \item This is the first item of an enumerated
        list that is nested within the itemized list.

      \item This is the second item of the inner list.
        \LaTeX\ allows you to nest lists deeper than
          you really should.
    \end{enumerate}
    This is the rest of the second item of the outer
    list.  It is no more interesting than any other
    part of the item.
  \item This is the third item of the list.
\end{itemize}
In a two-column layout, protracted indenting doesn't look very
good. Therefore, the Maps classfile provides \texttt{itemouter}- and
\texttt{enumouter} environments:
\begin{itemouter}
\item This is the first item of a non-indented itemized list,
  produced with the \texttt{itemouter} environment.
\item This is the second item.
\end{itemouter}
Now an enumerated version:
\begin{enumouter}
\item This is the first item of a non-indented enumerated list,
  produced with the \texttt{enumouter} environment.
\item This is the second item.
\end{enumouter}
And a version for descriptions:
\begin{descript}
\item[cow] A milk-producing animal that grazes grass and has
multiple stomachs
\item[kangoroo] An Australian hopping animal
\end{descript}

\section{Tabulars}

The Maps classfile adds some vertical space around horizontal rules
in tables. This makes vertical rules look funny, but most of the
time you are better off without vertical rules anyway; see table
\ref{tabulars}. If you really insist on vertical rules, use the
\texttt{deftables} document option.

\begin{table}[ht]

\begin{tabular}{|l|l|}
\hline
var & value\\
\hline
$Q_{s,\max}$ & 0.18\\
$K_{s}$      & 1.0\\
$Y_{x/s}$    & 0.5\\
$Y_{p/s}$    & 0.854\\
$Q_{p,\max}$ & 0.0045\\
$\mu_{\rm crit}$  & 0.01\\
$k_{h}$       & 0.002\\
$m_{s}$       & 0.025\\
\hline
\end{tabular}\hspace{2pc}
\begin{tabular}{@{}ll@{}}
\hline
var & value\\
\hline
$Q_{s,\max}$   & 0.18\\
$K_{s}$        & 1.0\\
$Y_{x/s}$      & 0.5\\
$Y_{p/s}$      & 0.854\\
$Q_{p,\max}$   & 0.0045\\
$\mu_{\rm crit}$ & 0.01\\
$k_{h}$        & 0.002\\
$m_{s}$        & 0.025\\
\hline
\end{tabular}
\caption{Tabulars with and without vertical rules}\label{tabulars}
\end{table}

\section{Wide typesetting in single-column layout}

For both single-column layouts, there are environments \texttt{fullwidth} and
\texttt{verboutdent} which typeset their content across the full page,
including most of the wide margin.

\begin{fullwidth}
x x x x x x x x x x x x x x x x x x x x x
x x x x x x x x x x x x x x x x x x x x x
x x x x x x x x x x x x x x x x x x x x x
x x x x x x x x x x x x x x x x x x x x x
\end{fullwidth}

\begin{verboutdent}
{}\/$xxxxxxxxxxxxxxxxxxxxxxxxxxxxxxxxxxxx
\end{verboutdent}
The implementation of \texttt{fullwidth} is rather simplistic and
may easily break, in which case more sophisticated hackery will be
needed.

\section{Fonts}

At production time, Computer Modern will be replaced with Bitstream
Charter, scaled to 95\%, and Latin Modern. For math, the eulervm
package will be used. You can safely ignore warnings about size
substitutions.

\section{Assembling your submission}

Please check whether all non-standard stylefiles and packages and all
non-standard fonts are included. We do have a current \TeX{} Live but,
although we do have access to CTAN, finding the right package by
name can occasionally be a challenge.

Avoid jpeg compression for screenshots. Conversion to pdf may
sometimes result in jpeg compression as well. Use \emph{e.g.} png
format instead.

Finally, a pdf of your article is appreciated. This way, we can
check more reliably whether your article compiles
correctly on our own systems.

\section{References}

If you have references, use whatever suits you. A few sample references:
see \cite{knuth}, or \cite{lamport}.

\theendnotes

\bibliographystyle{apalike}
\bibliography{maps}

%\begin{thebibliography}{}
%
%\bibitem[Knuth, 1986]{knuth}
%Knuth, D.~E. (1986).
%\newblock {\em The {\TeX{}}book}.
%\newblock Addison-Wesley Publishing Company.
%
%\bibitem[Kroonenberg, 2004]{mapsclass}
%Kroonenberg, S. (2004).
%\newblock The maps style.
%\newblock {\em Maps}, 30.
%
%\bibitem[Lamport, 1994]{lamport}
%Lamport, L. (1994).
%\newblock {\em {\LaTeX} a Document Preparation System}.
%\newblock Addison-Wesley Publishing Company, 2nd edition.
%
%\end{thebibliography}

\end{document}

%%%%%%%%%%%%%%%%%%%%%%%%%%%%%%%
% the bibtex file:

@BOOK{knuth,
author = "Donald E. Knuth",
title = "The {\TeX{}}book",
publisher = "Addison-Wesley Publishing Company",
year = 1986,
}

@BOOK{lamport,
author = "Leslie Lamport",
title = "{\LaTeX} a Document Preparation System",
publisher = "Addison-Wesley Publishing Company",
edition = "2nd",
year = 1994,
}

@ARTICLE{mapsclass,
author = "Siep Kroonenberg",
title = "The Maps style",
journal = "Maps",
volume = "30",
year = "2004"
}
% $Id$
